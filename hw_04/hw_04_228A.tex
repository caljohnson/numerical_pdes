\documentclass[12pt]{article}
\textwidth=17cm \oddsidemargin=-0.9cm \evensidemargin=-0.9cm
\textheight=23.7cm \topmargin=-1.7cm

\usepackage{amssymb, amsmath, amsfonts}
\usepackage{moreverb}
\usepackage{graphicx}
\usepackage{enumerate}
\usepackage{graphics}
\usepackage{color}
\usepackage{array}
\usepackage{float}
\usepackage{hyperref}
\usepackage{textcomp}
\usepackage{alltt}
\usepackage{physics}
\usepackage{mathtools}
\usepackage{tikz}
\usetikzlibrary{positioning}
\usetikzlibrary{arrows}
\usepackage{pgfplots}
\usepackage{bigints}
\usepackage[utf8]{inputenc}
\usepackage[english]{babel}
\usepackage{amsthm}
\usepackage{fancyhdr}
\usepackage[makeroom]{cancel}
\pagestyle{fancy}
\allowdisplaybreaks

\newcommand{\E}{\varepsilon}

\newcommand{\suchthat}{\, \mid \,}
\newcommand{\ol}[1]{\overline{#1}}
\newcommand{\bbar}[1]{\overline{#1}}
\newcommand{\inpd}[1]{{\left< \, #1 \, \right>}}
\renewcommand{\theenumi}{\alph{enumi}}
\newcommand\Wider[2][3em]{%
\makebox[\linewidth][c]{%
  \begin{minipage}{\dimexpr\textwidth+#1\relax}
  \raggedright#2
  \end{minipage}%
  }%
}

\def\R{\mathbb{R}}
\def\C{\mathbb{C}}
\def\H{\mathcal{H}}
\DeclareMathOperator*{\esssup}{\text{ess~sup}}
\newcommand{\resolv}[1]{\rho(#1)}
\newcommand{\spec}[1]{\sigma(#1)}
\newcommand{\iffR}{\noindent \underline{$\Longrightarrow$:} }
\newcommand{\iffL}{\noindent \underline{$\Longleftarrow$:} }
\newcommand{\lightning}{\textbf{\Huge \Lightning}}
\newcommand{\spt}[1]{\text{spt}(#1)}
\def\ran{\text{ ran}}
   
\newenvironment{myprob}[1]
    {%before text commands
    %{\Huge \_ \_ \_ \_ \_ \_ \_ \_ \_ \_ \_ \_ \_ \_ \_ \_ \_ \_ } \\
    \noindent{\Huge$\ulcorner$}\textbf{#1.}\begin{em}
    }
    { 
    %after text commands
    \end{em} \\ \hphantom{l} \hfill {\Huge$\lrcorner$} }
%	{\noindent \rule{7.5cm}{2pt} \textgoth{#1} \rule{8.cm}{2pt} \begin{em}}
%	{\end{em}\\ \vspace{0.1pt}\noindent \rule{\textwidth}{2pt}}
%
\setcounter{section}{-1}




\begin{document}
\lhead{MATH228A}
\chead{Carter Johnson - Homework 04}
\rhead{\today}

{\let\newpage\relax} 


%%%%%%%%%%%%%%%%%%%%%%%%%%%%%%%%%%%%%%%%%%%%%%%%%%%%% P1
\begin{myprob}{Problem 1}
Write a multigrid V-cycle code to solve the Poisson equation in two dimensions on the unit square with Dirichlet boundary conditions. Use full weighting for restriction, bilinear interpolation for prolongation, and red-black Gauss-Seidel for smoothing.
Use this code to solve
$$\Delta u = -\exp\qty( - (x-0.25)^2 - (y-0.6)^2)$$
on the unit square $(0, 1) \times (0, 1)$ with homogeneous Dirichlet boundary conditionss for different grid spacings. How many steps of pre and postsmoothing did you use? What tolerance did you use? How many cycles did it take to converge? Compare the amount of work needed to reach convergence with your solvers from Homework 3 taking into account how much work is involved in a V-cycle.
\end{myprob}

For the given problem, I ran my multigrid V-cycle code with the stopping criterion
max-norm of the residual relative to the max-norm of the RHS function, with a tolerance of $10^{-10}$. Using my analysis from 2, I found that (1,1), (2,1), and (2,2) were top contenders for optimal pre-smoothing, post-smoothing step numbers, and the fastest results were obtained for 2 pre-smoothing steps and 1 post-smoothing step. The results show that iteration count for the multigrid algorithm is grid-independent. The choice of $(\nu_1, \nu_2) = (1,1)$ performs slightly better than $(2,1)$ and $(2,2)$ on most grid sizes, and is the fastest solver on the finest grid. The results are tabulated as follows: \\
For $(\nu_1, \nu_2) = (1,1)$,  \\
\begin{center}
\begin{tabular}{||c|c|c||}
\hline \hline
   grid spacing $h$ &   iteration count &   run time (seconds) \\
\hline \hline
       $2^{-5}$    &  12 &  0.296257 \\
       $2^{-6}$   &   12 &  1.23618    \\
       $2^{-7}$  &  12 &  4.7184  \\
       $2^{-8}$ &  12 & 18.9813   \\
       $2^{-9}$ & 12 & 77.6296 \\
\hline \hline
\end{tabular}
\end{center}

\noindent For $(\nu_1, \nu_2) = (2,1)$,  \\
\begin{center}
\begin{tabular}{||c|c|c||}
\hline \hline
   grid spacing $h$ &   iteration count &   run time (seconds) \\
\hline \hline
       $2^{-5}$   &                10 &             0.303806 \\
       $2^{-6}$   &                10 &             1.24774   \\
       $2^{-7}$  &                10 &             4.95247  \\
       $2^{-8}$ &                10 &            19.4882   \\
       $2^{-9}$ & 10 & 78.2237 \\
\hline \hline
\end{tabular}
\end{center}

\noindent For $(\nu_1, \nu_2) = (2,2)$,  \\
\begin{center}
\begin{tabular}{||c|c|c||}
\hline \hline
   grid spacing $h$ &   iteration count &   run time (seconds) \\
\hline \hline
       $2^{-5}$    &                 9 &             0.331599 \\
       $2^{-6}$   &                 9 &           1.37659  \\
       $2^{-7}$ &                 9 &             5.52204  \\
       $2^{-8}$ &                 9 &            21.5597   \\
       $2^{-9}$ & 9 & 88.6846 \\
\hline \hline
\end{tabular}
\end{center}

Compared to the solvers from homework 3, i.e., the Jacobi, GS-lex, and SOR iterative methods, this program takes significantly fewer total iterations, and considerably less work overall. Each iteration of those methods is the same amount of work as a single smoothing operation in the multigrid V-cycle, call this amount of work $W$.  In each V-cycle iteration of the multigrid algorithm, we do $\nu$ smoothing operations on the finest mesh. We also perform a single residual computation, restriction opeeration, and interpolation operation, each of which is comparable work to a smoothing operation, so the total work on the finest mesh is thus $(\nu + 4)W$. As we discussed in class, every coarser mesh level requires a factor of 4 less work, so the total work of all the smooths in the limit of many mesh levels is 
$$\sum_{\ell =1}^L (\nu + 3) W \qty(\dfrac{1}{4})^{\ell - 1} \approx (\nu+3)W\qty(\dfrac{1}{1-\frac{1}{4}}) = \dfrac{4}{3}(\nu+3)W.$$

So for a $64\times64$ mesh ($h=2^{-6}$) and tolerance of $10^{-6}$, multigrid clearly outperforms the best solver from homework 3, SOR.  Note that MG with (1,1) as the pre-smooth, post-smooth steps takes slightly less work than the (2,1) and (2,2) counterparts.
\begin{center}
\begin{tabular}{||c|c|c|c||}
\hline \hline
 & iterations & work per iteration & total work \\
\hline \hline
SOR & 144 & W & 144 W \\
MG (1,1) & 8 & $\frac{4}{3}(2+4)W = 8W$ & 48W \\
MG (2,1) & 6 & $\frac{4}{3}(3+4)W = \frac{28}{3}W$ & 56W \\
MG (2,2) & 6 & $\frac{4}{3}(4+4)W = \frac{32}{3}W$ & 64 W\\
\hline \hline
\end{tabular}
\end{center}

And for a $264\times264$ mesh ($h=2^{-8}$) and tolerance of $10^{-7}$, multigrid blows SOR out of the water. Note that MG with (1,1) pre-smooth, post-smooth steps still takes the least amount of work.
\begin{center}
\begin{tabular}{||c|c|c|c||}
\hline \hline
 & iterations & work per iteration & total work \\
\hline \hline
SOR & 658 & W & 658 W \\
MG (1,1) & 9 & $8W$ & 72W \\
MG (2,1) & 7 & $\frac{28}{3}W$ & 65.33 W \\
MG (2,2) & 7 & $\frac{32}{3}W$ & 74.7 W\\
\hline \hline
\end{tabular}
\end{center}

\newpage
\begin{verbatim}
#MAT228A HW 4
#Multigrid V-Cycle Solver

#Multigrid V-cycle with
#full-weighting, bilinear interpol, GS-RB smoothing
#for Poission equation with Dirichlet BCs 


from __future__ import division

import numpy as np
from math import exp, sin, pi

from GS_RB_smoother import GS_RB_smoother
from full_weighting_restriction import full_weighting_restriction
from bilinear_interpolation import bilinear_interpolation
from compute_residual import compute_residual
from direct_solve import trivial_direct_solve
  
def V_cycle(u, f, h, v1, v2):
  #presmooth v1 times
  u = GS_RB_smoother(u,f, h, v1)

  #compute residual
  res = compute_residual(u, f, h)

  #restrict residual
  res2 = full_weighting_restriction(res, h)

  #solve for coarse grid error, check grid level to decide whether to solve or be recursive
  if h == 2**(-2):
    error = trivial_direct_solve(res2, 2*h)
  else:
    error = np.zeros((int(1/(2*h)+1), int(1/(2*h)+1)), dtype=float)
    error = V_cycle(error, res2, 2*h, v1, v2) 

  #interpolate error
  error2 = bilinear_interpolation(error, 2*h)

  #correct (add error back in)
  u = u+error2

  #post-smooth v2 times
  u = GS_RB_smoother(u, f, h, v2)

  return u
\end{verbatim}
\newpage
\begin{verbatim}
#MAT228A Homework 4
#GS-RB Smoother

#Use GS-RB to smooth 
#in larger multigrid V-cycle program
#to find soluton to Poisson eqn.
#with homogeneous Dirichlet BC's

from __future__ import division

import numpy as np

def GS_RB_smoother(u, f, h, steps):

  #set number of grid points in each row/column
  n = int(1/h - 1)
  
  #separate red, black indices into two lists
  reds = []
  blacks = []
  for i in range(1,n+1):
    for j in range(1,n+1):
      if (i+j)%2==0:
        reds.append((i,j))
      else:
        blacks.append((i,j))

  #begin iterative scheme
  for k in range(steps):

    #loop red 
    for (i,j) in reds:
      # print "red", i,j
      u[i][j] = (1/4)*(u[i-1][j]+u[i][j-1]+u[i+1][j] + u[i][j+1] - (h**2)*f[i][j])
    
    #loop black 
    for (i,j) in blacks:
      # print "black", i,j
      u[i][j] = (1/4)*(u[i-1][j]+u[i][j-1]+u[i+1][j] + u[i][j+1] - (h**2)*f[i][j])

  return u
\end{verbatim}

\newpage

\begin{verbatim}

\end{verbatim}

%%%%%%%%%%%%%%%%%%%%%%%%%%%%%%%%%%%%%%%%%%%%%%%%%%%%% P2
\begin{myprob}{Problem 2}
Numerically estimate the average convergence factor,
$$ E_k = \qty(\dfrac{\norm{e^{(k)}}_\infty}{\norm{e^{(k)}}_\infty})^{1/k},$$
for different numbers of presmoothing steps, $\nu_1$, and postsmoothing steps, $\nu_2$, for $\nu = \nu_1 + \nu_2 \leq 4$. Be sure to use a small value of $k$ because convergence may be reached very quickly. What test problem did you use? Do your results depend on the grid spacing? Report the results in a table, and discuss which choices of $\nu_1$ and $\nu_2$ give the most efficient solver. \\
\end{myprob} \\
I used my multigrid program from problem 1 to solve the problem
$$
\Delta u = -2 \sin(\pi x) \sin(\pi y)
$$
on the unit square $(0, 1)\times(0, 1)$ with homogeneous Dirichlet boundary conditions, which has the known solution $$u(x,y) = \sin(\pi x) \sin(\pi y).$$
I performed an analysis of all the different pairings $(\nu_1, \nu_2)$ for grid spacings $h=2^{-5}, 2^{-6}, 2^{-7}$ with stopping criterion relative iterate differences with tolerance $10^{-6}$, and these all achieved relatively similar results. The results do not depend on the grid spacing, as the data will testify. I report the average convergence factors for 1-5 iterations (e.g., $E_3$ is the average convergence factor among 3 iterations, while $E_5$ is the average convergence factor among 5 iterations), and 5 was chosen as the largest $k$ to consider since my lowest reported iteration count for a multigrid solve was 8.  
The following table is for $h=2^{-5}$, tolerance $10^{-10}$.\\
\begin{center}
\begin{tabular}{||c|cccc|c | c||}
\hline \hline
 ($\nu_1, \nu_2$)   & $E_1$ & $E_2$ &  $E_3$ & $E_4$ &   iterations & run times\\
\hline \hline
  \color{red}(0, 1)   &   \color{red}0.305393  &  \color{red}0.301592  &  \color{red}0.297944  &   \color{red}0.291537 &   \color{red}20 & \color{red}0.519646\\
 (1, 0)   &  0.293605  &      0.288987  &      0.291686  &       0.291592 &           24 & 0.68032 \\ \hline
  \color{red}(1, 1)   &   \color{red}0.11991   &  \color{red}0.116894  &  \color{red}0.0975584 &  \color{red}0.156273 &  \color{red}12 & \color{red}0.404654\\
 (0, 2)   &  0.179352  &      0.176355  &      0.1692    &       0.118281 &           14 & 0.515413 \\ 
 (2, 0)   &  0.193867  &      0.186207  &      0.180232  &       0.149008 &           16 & 0.589619 \\ \hline
  \color{red}(1, 2)   &   \color{red}0.0805982 &       \color{red}0.0760598 &      0.0647618 &       0.166059 &    \color{red}10 & \color{red}0.390981\\
  \color{red}(2, 1)   &  0.0809173 &      0.0763635 &       \color{red}0.064411  &        \color{red}0.166031 &   \color{red}10 & 0.405133 \\
 (3, 0)   &  0.140947  &      0.131789  &      0.117586  &       0.162013 &           13 & 0.514157 \\
 (0, 3)   &  0.121375  &      0.117422  &      0.0979251 &       0.156212 &           12 & 0.471079 \\ \hline
  \color{red}(2, 2)   &  0.0606403 &      0.0543182 &  0.0831176 &       0.167628 &  \color{red}9 & 0.41648  \\
  \color{red}(1, 3)   &  \color{red}0.0606286 &  \color{red}0.0543059 &      0.0831232 &       0.167629 &  \color{red}9  & \color{red}0.407873\\
  \color{red}(3, 1)   &  0.0607858 &      0.0544584 &   \color{red}0.0830569 &     \color{red}0.167623 &   \color{red}9  & 0.443\\
 (4, 0)   &  0.109314  &      0.100546  &      0.0728343 &       0.166327 &           12 &0.590817 \\
 (0, 4)   &  0.0914259 &      0.0863728 &      0.0404893 &       0.164779 &           10 & 0.490951\\
\hline \hline
\end{tabular}
\end{center}
 
For $h=2^{-6}$, tolerance $10^{-10}$,
\begin{center}
\begin{tabular}{||c|cccc|c|c||}
\hline \hline
 ($\nu_1, \nu_2$)   & $E_1$ & $E_2$ &  $E_3$ & $E_4$ &   iterations & run times \\
\hline \hline
   \color{red}(0, 1)   &    \color{red}0.305759  &   \color{red}0.303893  &   \color{red}0.302047  &   \color{red}0.29956   &   \color{red}20 & \color{red}1.60016 \\
 (1, 0)   &  0.294261  &      0.295252  &      0.298011  &      0.300046  &           26 & 2.12789\\ \hline
   \color{red}(1, 1)   &   \color{red}0.12071   & \color{red}0.119945  & \color{red}0.115927  & \color{red}0.0577091 & \color{red}12 & \color{red}1.3413\\
 (0, 2)   &  0.180233  &      0.179461  &      0.177762  &      0.170408  &           14  & 1.42908 \\
 (2, 0)   &  0.194353  &      0.191018  &      0.188791  &      0.182702  &           17 & 1.83374 \\ \hline
 \color{red}(1, 2)   & \color{red}0.0815659 & \color{red}0.0804702 & \color{red}0.0700874 &      0.111815  & \color{red}10 & \color{red}1.3723 \\
 \color{red}(2, 1)   &  0.0818544 &      0.0807546 &      0.070458  & \color{red}0.11171   & \color{red}10 & 1.37951 \\
 (3, 0)   &  0.141378  &      0.137511  &      0.133371  &      0.111955  &           14 & 1.91811 \\
 (0, 3)   &  0.122455  &      0.121482  &      0.117529  &      0.0682134 &           12  & 1.613 \\ \hline
 \color{red}(2, 2)   &  0.0616633 &      0.0601744 &      0.0327965 &      0.11682   &  \color{red}9 & 1.46634\\
 \color{red}(1, 3)   &  0.0616527 &      0.0601636 &      0.0327591 &      0.116821  &  \color{red}9 & 1.42421\\
 \color{red}(3, 1)   &  0.061792  &      0.0603017 &      0.0332196 &      0.116802  &  \color{red}9 & \color{red}1.41787 \\
 (4, 0)   &  0.109726  &      0.106782  &      0.100253  &      0.113894  &           13 & 2.00789\\
 (0, 4)   &  0.0926053 &      0.0913771 &      0.0837286 &      0.106415  &           10 & 1.51836\\
\hline \hline
\end{tabular}
\end{center}

For $h=2^{-7}$, tolerance $10^{-10}$: \\
\begin{center}
\begin{tabular}{||c|cccc|c||}
\hline \hline
 ($\nu_1, \nu_2$)   & $E_1$ & $E_2$ &  $E_3$ & $E_4$ &   iterations \\
\hline \hline
 \color{red}(0, 1)   &  0.305848  &      0.304463  &      0.303061  &      0.301479  &  \color{red}21 \\
 (1, 0)   &  0.294425  &      0.297873  &      0.300711  &      0.302654  &           27 \\ \hline
 \color{red}(1, 1)   &  0.120909  &      0.120695  &      0.119705  &      0.112979  & \color{red}12 \\
 (0, 2)   &  0.180453  &      0.180232  &      0.179786  &      0.178067  &           14 \\
 (2, 0)   &  0.194475  &      0.192975  &      0.192209  &      0.191775  &           18 \\ \hline
 \color{red}(1, 2)   &  0.0818076 &      0.0815357 &      0.0792587 &      0.0481262 & \color{red}10 \\
 \color{red}(2, 1)   &  0.0820882 &      0.0818156 &      0.0795544 &      0.0470683 & \color{red}10 \\
 (3, 0)   &  0.141487  &      0.139291  &      0.13751   &      0.133174  &           15 \\
 (0, 3)   &  0.122724  &      0.122481  &      0.121534  &      0.115184  &           12 \\ \hline
 \color{red}(2, 2)   &  0.0619189 &      0.0615518 &      0.0572464 &      0.0771726 & \color{red}9 \\
 \color{red}(1, 3)   &  0.0619085 &      0.0615412 &      0.057234  &      0.0771781 &  \color{red}9 \\
 \color{red}(3, 1)   &  0.0620431 &      0.0616766 &      0.0573914 &      0.0771083 &  \color{red}9 \\
 (4, 0)   &  0.10983   &      0.10803   &      0.10601   &      0.095701  &           13 \\
 (0, 4)   &  0.0928997 &      0.0925947 &      0.0908535 &      0.0700194 &           10 \\
\hline \hline
\end{tabular}
\end{center}

For $h=2^{-8}$, tolerance $10^{-10}$,
\begin{center}
\begin{tabular}{||c|cccc|c|c||}
\hline \hline
 ($\nu_1, \nu_2$)   & $E_1$ & $E_2$ &  $E_3$ & $E_4$ &   iterations & run times\\
\hline \hline
 (0, 1)   &  0.305871  &      0.304606  &      0.303314  &      0.301955  &           21 &     28.4614 \\
 (1, 0)   &  0.294466  &      0.29897   &      0.302007  &      0.304028  &           29 &     38.5927 \\ \hline
 \color{red}(1, 1)   &  0.120959  &      0.120882  &      0.120614  &      0.119052  &           12 &     \color{red}21.1899 \\
 (0, 2)   &  0.180508  &      0.180424  &      0.180285  &      0.179836  &           14 &     25.349  \\
 (2, 0)   &  0.194505  &      0.193764  &      0.193112  &      0.193811  &           19 &     34.2363 \\ \hline
 (1, 2)   &  0.0818681 &      0.0817999 &      0.081246  &      0.0754371 &           10 &     21.9906 \\
 (2, 1)   &  0.0821466 &      0.0820787 &      0.081529  &      0.0757927 &           10 &     22.4021 \\
 (3, 0)   &  0.141514  &      0.139864  &      0.139102  &      0.138534  &           16 &     35.8415 \\
 (0, 3)   &  0.122792  &      0.12273   &      0.122495  &      0.121037  &           12 &     26.4578 \\ \hline
 (2, 2)   &  0.0619828 &      0.0618914 &      0.0608837 &      0.038593  &            9 &     24.3085 \\
 (1, 3)   &  0.0619724 &      0.0618809 &      0.0608727 &      0.0385485 &            9 &     24.4019 \\
 (3, 1)   &  0.0621058 &      0.0620156 &      0.0610131 &      0.0391059 &            9 &     23.787  \\
 (4, 0)   &  0.109856  &      0.108447  &      0.107895  &      0.105943  &           14 &     37.0952 \\
 (0, 4)   &  0.0929733 &      0.0928971 &      0.0924709 &      0.088774  &           10 &     26.4858 \\
\hline \hline
\end{tabular}
\end{center}

\end{document}